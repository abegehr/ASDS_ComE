%% temporarily setting to anonymous for blind peer-reviews
%\documentclass[sigconf, anonymous]{acmart}
\documentclass[sigconf]{acmart}

\usepackage{hyperref}

\AtBeginDocument{
  \providecommand\BibTeX{{
    \normalfont B\kern-0.5em{\scshape i\kern-0.25em b}\kern-0.8em\TeX}}}

\setcopyright{none}
\copyrightyear{}
\acmDOI{}
\acmISBN{}
\acmConference[Data Science Seminar]{Data Science Seminar}{2019}{Uni Passau}


\begin{document}

\title{[Experiment] Learning Community Embedding with Community Detection and Node Embedding on Graph (CD)}

\author{Anton Begehr}
\affiliation{
  \institution{University of Passau}
  \streetaddress{Innstraße 33}
  \city{Passau}
  \country{Germany}}
\email{a.begehr@fu-berlin.de}


\begin{abstract}
  A clear and well-documented \LaTeX\ document is presented as an
  article formatted for publication by ACM in a conference proceedings
  or journal publication. Based on the ``acmart'' document class, this
  article presents and explains many of the common variations, as well
  as many of the formatting elements an author may use in the
  preparation of the documentation of their work.
\end{abstract}


\maketitle

\section{Introduction}

In their \citeyear{Cav17} paper \textit{Learning Community Embedding with Community Detection and Node Embedding on Graph} the authors \citeauthor{Cav17} explore graph embeddings by utilizing a three step closed loop optimization process consisting of Community Detection, Community Embedding, and Node Embedding: ComE.\cite{Cav17} They then apply ComE and further graph embedding algorithms on excerpts of multiple, well known graph datasets: Karate Club, BlogCatalog, Flicker, Wikipedia, DBLP. They choose DeepWalk/SF, Line, Node2Vec, GraRep, and M-NMF to measure the quality of ComE's results using Micro-F1 and Macro-F1 for classification results and conductance and normalized mutual information (NMI) for the resulting graph embeddings. It is worthwhile to note, that ComE works with elementary graphs, meaning graphs consisting of nodes and edges, disregarding, for example, node properties, edge properties, and edge weights.

In this paper we will explore the graph embedding algorithm ComE developed by \citeauthor{Cav17}, generate embeddings using ComE for the Twitter dataset \cite{TwitterData} crawled by \citeauthor{TwitterData}, and compare the results to communities gained by applying Louvain Modularity using normalized mutual information (NMI) as a comparison measure.


\section{Twitter Data}

\subsection{ComE Data Requirements}

The ComE algorithm requires the graph data to be supplied as a sparse adjacency matrix. The example code uses a MATLAB .mat file import. The data supplied by \citeauthor{TwitterData} is in an edge list format inside a CSV file. Each row represents an edge from one node to the other.

ComE also requires the nodes to be labeled for knowing the number K of clusters it should optimize for and testing the resulting clusters against the supplied ground truth.

\subsection{Data Preparation}

These two requirements ComE has



\bibliographystyle{ACM-Reference-Format}
\bibliography{sources}


\begin{acks}
To fill.
\end{acks}


\appendix

\section{Code}

\subsection{Generate graph}

Lorem ipsum dolor sit amet, consectetur adipiscing elit. Morbi
malesuada, quam in pulvinar varius, metus nunc fermentum urna, id
sollicitudin purus odio sit amet enim. Aliquam ullamcorper eu ipsum
vel mollis. Curabitur quis dictum nisl. Phasellus vel semper risus, et
lacinia dolor. Integer ultricies commodo sem nec semper.

\subsection{Generate Labels}

Etiam commodo feugiat nisl pulvinar pellentesque. Etiam auctor sodales
ligula, non varius nibh pulvinar semper. Suspendisse nec lectus non
ipsum convallis congue hendrerit vitae sapien. Donec at laoreet
eros. Vivamus non purus placerat, scelerisque diam eu, cursus
ante. Etiam aliquam tortor auctor efficitur mattis.

\subsection{Read CSVs}

Nam id fermentum dui. Suspendisse sagittis tortor a nulla mollis, in
pulvinar ex pretium. Sed interdum orci quis metus euismod, et sagittis
enim maximus. Vestibulum gravida massa ut felis suscipit
congue. Quisque mattis elit a risus ultrices commodo venenatis eget
dui. Etiam sagittis eleifend elementum.

Nam interdum magna at lectus dignissim, ac dignissim lorem
rhoncus. Maecenas eu arcu ac neque placerat aliquam. Nunc pulvinar
massa et mattis lacinia.


\end{document}
\endinput
